\begin{table}
    \centering
    \begin{adjustbox}{width=\textwidth}
    \begin{tabular}{|p{3.5cm}|p{6cm}|p{7cm}|}
        \hline
        \textbf{Pollutants} & \textbf{Source of Production} & \textbf{Effect on Humans and Environment} \\
        \hline
        Oxides of Carbon ($CO_X$): & Combustion of fossil fuels & $CO_2$ is a major contributor to green house effect, \\
        Carbon dioxide ($CO_2$) & Oil and other energy production and transport fuels & $CO_2$ forms weak carbonic acid that adds to acidic rain, \\
        Carbon monoxide ($CO$) & Burning of biomass & $CO$ binds to haemoglobin, which can cause "asphyxia" in humans. \\
        \hline

        Oxides of Sulphur ($SO_X$): & Combustion fossil fuels e.g., coal, petroleum; & $SO_2$ It has maximum deleterious effects as it damages the human lungs, \\
        Sulphur dioxide ($SO_2$) & paper manufacturing; municipal incineration; & Contribute to acid rain; \\
        Sulphur tri-oxide ($SO_3$) & smelting of ore for extraction of metal &  Causes corrosion of paint, metals, and injury or death to animals and plants. \\
        Sulphate ($SO_4$) & & \\
        \hline

        Oxides of Nitrogen ($NO_X$): & Burning of fuels; & Constitutes secondary pollutants: peroxy acetyl nitrate (PAN) and nitric acid($HNO_3$); \\
        Nitrogen oxide ($NO$) & Burning of biomass; & Suppresses growth of plants and causes damage to tissue; \\
        Nitrogen dioxide ($NO_2$) & by-product in the manufacture of fertilizers & causes eye irritation, viral infections such as influenza; \\
        Nitrous oxide ($N_2O$) & &  \\
        Nitrate ($NO_3$) & & Nitrate in the atmosphere impairs visibility. \\
        \hline

        Suspended Particulate Matter(SPM) & Combustion of fuels; construction of buildings; mining; thermal power stations; & Have chronic effects on the respiratory system, deposition on the surface of green leaves interferes with the absorption of $C0_2$ and release of $0_2$; \\
       Dust, soot, silica, asbestos, liquid spray, mist, etc. & Stone crushing; forest fires; incineration of waste.& Blocking sunlight; particles ranging in size from 0.1 to 10um, cause the greatest damage to the lungs. \\
        \hline

        Photo-chemical oxidants & Photo-chemical reactions in the atmosphere involving sunlight, nitrogen oxides and hydrocarbons & Produces haze; irritation of the eyes, nose , and throat; \\
        Ozone ($O_3$) & &  Respiratory problems; sunlight blocking; \\
        \hline
    \end{tabular}
    \end{adjustbox}
    \caption{Sources and Effect on Humans and Environment of some common air pollutants}
\end{table}